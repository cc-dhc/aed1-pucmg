\documentclass[12pt]{article}

\usepackage[portuguese]{babel}
\usepackage{csquotes}
\usepackage{indentfirst}
\usepackage{biblatex}
\addbibresource{artigo.bib}

\title{Analise do artigo}

\author{Pedro Francisco Staino Santayana}

\date{}

\begin{document}

\maketitle

\section*{Ideia geral do artigo}
O artigo\cite{inproceedings} tem como finalidade analisar as linguagens de programação Java e C/C++
com relação ao tempo de execução de alguns algoritmos. Esses algoritmos tinham
como objetivo emular estresse à pilha de execução, à memória primária, 
à memória secundária e à unidade lógica e aritmética. Com isso, o estudo fez uma comparação
entre uma linguagem compilada e uma linguagem interpretada mas que utiliza compilação dinâmica.

\section*{Problema estudado e solução}

Foi dado um foco maior na metodologia dos testes aplicados. Foram realizados os mesmos
testes nos compiladores GCC e JDK, com uma versão recente e uma defasada de cada um.
Também usaram optimizações durante a fase de compilação, versões anteriores de cada compilador
e as melhores implementações possíveis de cada algoritmo.

Também foi utilizado o Projeto Fatorial $2^kr$, que analisa os efeitos de $k$ fatores,
cada qual com dois níveis. Sendo os fatores: compilação customizada e 
versões do compilador. Java mostrou uma evolução com relação ao compilador com o passar do tempo,
enquanto a otimização do compilador mostrou-se necessária para a linguagem C/C++. 

\section*{Análise dos trabalhos relacionados}
O artigo consegue passar informações prévias sobre o problema durante todas as partes
do texto, mas principalmente em \textit{Trabalhos Relacionados}. Essa sessão apresentou,
de forma resumida, algumas pesquisas que trazem resultados muito interessantes. Uma dessas, relacionada a optimização da linguagem Java,
conclui que é possível que a linguagem tenha um melhor desempenho em comparação a C/C++
por causa da compilação dinâmica.

\section*{Pontos fortes e pontos fracos}
O questionamento sobre a eficiência da linguagem Java em relação a linguagens compiladas como C/C++
foi o ponto de partida do artigo, e por isso acredito que a abordagem metodologica
serviu para responder, de forma empírica, a questão. Mas também acredito que faltou uma abordagem mais
aprofundada com relação ao resultado do algoritmo de Busca Sequencial, em que Java obteve uma pequena vantagem
com relação a C/C++.

Uma das justificativas é a compilação dinâmica, que permite um ganho de performance
porque parte do código é compilado e recompilado durante
a execução do programa, trazendo certas optimizações que não estão
disponíveis durante a compilação estática em troca de um período maior de inicialização.

Em alguns casos, parte da compilação dinâmica é feita em momentos posteriores a inicialização, 
causando alguns travamentos a mais. Mas, mesmo assim, o compilador Just-in-Time
ainda assim é uma ferramenta essencial para a performance de programas Java. O tema realmente merece uma pesquisa à parte.



\printbibliography

\end{document}